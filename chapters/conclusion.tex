\chapter{Conclusion}
\label{chap:conclusion}

As conclusion of the gathered information about \gls{tls}, its use, the comparison of the new and the prior version and with the experiences of the migration from the older to the newer version, the following can be state:

\gls{tls} 1.3 has significant changes with lots of simplifications, which results in a improved security and better performance during communications and transactions. 
\begin{itemize}
   \item PFS: \gls{tls} 1.3 always uses perfect forward secrecy (PFS), which means that each session is protected separately, even if any transmission is interrupted. So, there will be no common key. 

   \item 0-\gls{rtt}: The entire round-trip can be avoided because the client now has the possibility to reconnect to a server to which it has been connected before. This will improve the load time.

   \item handshake: The full handshake in \gls{tls} 1.2 costs time, increases server load and causes connection latency. This is now minimized.

   \item  Strong ciphers: \gls{tls} 1.3 makes only use of the strong ciphers (AES-GCM and ChaCha-Poly), therefore it will be easier to configure applications that are already secured by default.
\end{itemize}
Therefore it is very useful and important to update to \gls{tls} 1.3.


