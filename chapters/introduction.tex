\chapter{Introduction}
\label{chap:introduction}

Transport Layer Security (\gls{tls}) Protocol is a cryptographic protocol that provides secure transport connection between applications over a computer network, e.g. web server and web browser. 
Using \gls{tls} prevents eavesdropping, tampering and message forgery. It provides privacy and data integrity between communicating applications. The protocol secures transmitted data using encryption. Secure communication is enabled with Authenticated Key Etablishment (\gls{ake}) and secure channel with Authenticated Encryption (\gls{ae}). 
Using Datagram \gls{tls} (\gls{dtls}) the stateless, \gls{tls} is feasible. It offers server and, optionally, client authentication to confirm the identities of parties involved in the communication. 
 
Furthermore, the integrity check value implemented in the protocol provides integrity for the transferred data. \gls{tls} is also known as Secure Socket Layer (\gls{ssl}), its predecessor. 
 
To use \gls{tls} all communicators must know that the other is supporting \gls{tls}.

\gls{tls} 1.2 was defined in \gls{rfc}5246 in August 2008. Its successor \gls{tls} 1.3 is defined in \gls{rfc}8446 in August 2018.
 \cite{RFC5246}\cite{ms:overview}

\section{Structure of the \gls{tls} Protocol}
\label{sec:stucture}

The \gls{tls} Protocol is layered between the Application layer and the \gls{tcp}/\gls{ip} layer according to the Internet Model (or between Session and Transport layers according to the \gls{osi} Model) where it can secure and then send application data to the transport layer \cite{ms:overview}. Thus it can support multiple application layer protocols such as \gls{http}, \gls{ftp}, \gls{smtp} or \gls{pop}. The protocols using \gls{tls} become respectively \gls{https}, \gls{ftps}, \gls{smtps}, \gls{pops} and so on.

The \gls{tls} Protocol can be split into two layers. The lower layer of the \gls{tls} is the Record Protocol. The upper layer is the Handshake layer which consists of the following protocols: Handshake Protocol, ChangeCipherSpec Protocol and Alert Protocol. ChangeCipherSpec has been removed from the 1.3 version. Figure \ref{fig:tls_structure} illustrates the structure of the \gls{tls} Protocol. \cite{tlsstrukt}

\begin{figure}[H]
	\centering
		\includegraphics[scale=1]{images/tls_structure.jpg}
	\caption{\gls{tls} 1.2 Protocol Structure \cite{ms:overview}}
	\label{fig:tls_structure}
\end{figure}

\section{Handshake Protocol}
\label{sec:handshake_protocol}
 A handshake is an automated process of negotiation between communicators. The \gls{tls} Handshake Protocol is the most widely-used \gls{ake} Protocol to negotiate session information between the client and the server. 
 After the establishment of the \gls{tcp} connection between a client and a server the handshake protocol will be triggered. 
 Thus, the purpose of this protocol is to negotiate and define these parameters:
 
 \begin{itemize}
\item to authenticate the server and optionally also the client
 \item to agree on the version of the \gls{tls}
 \item to have an agreement of the cipher suites
 \item to arrange with the extension
 \item to derive authenticated encryption key for the connection
 \item if required, to verify the certificates
 \item finally all these points have to be assured between each communicators
\end{itemize}

To protect data, these parameters are then used by the record layer. As mentioned, usually it is only the server which is authenticated, but once a secure channel is established, the mutual authentications may also be carried out. This is permitted with the use of renegotiation but requires a public key infrastructure (\gls{pki}). 
\cite{ms:overview}
\cite{ms:handshake}

\section{ChangeCipherSpec Protocol}
\label{sec:changeciphfer_protocol}
This protocol is responsible for informing about a change of one set of keys to another. It means that a transition in ciphering would be signaled through this protocol. This message is either sent by the server or the client. 

This protocol triggers an instruction to the record layer.

The keying material is raw data which is used for encryption between the client and server. On the basis of the exchanged information from the handshake protocol, the keys will be then computed.
The ChangeCipherSpec Protocol consists of a byte with value 1.
The ChangeCipherSpec Protocol also includes a message to inform other parties in the \gls{ssl}/\gls{tls} session about the change.   \cite{ms:overview}

The ChangeCipherSpec protocol is removed from the 1.3 version of the \gls{tls} protocol.
\cite{WikipediaCipher}
\section{Alert Protocol}
\label{sec:alert_protocol}
The peer will be notified by the alert messages about the change in status or error condition (fatal/warning).These messages are also compressed and encrypted. If a fatal error occurs, the session will close immediately.
Fatal:
\begin{itemize}
	\item unexpected\_message
	 \item bad\_record\_\gls{mac}
	 \item handshake\_failure 
	 \item etc.
\end{itemize}
	
Warning:
\begin{itemize}
\item close\_notify
\item unsupported\_certificates
\item certificate\_expired
\item etc.

\end{itemize}

The full list of the alerts to notify the peer of normal or error condition is declared in \gls{rfc}2246 \cite{rfc2246}. All these alerts consist of 2 bytes. The first byte defines the kind (eg. fatal or warning) and the second specifies what happened \cite{W.Stalling} \cite{ms:overview}

\section{Record Protocol}
\label{sec:record_protocol}

The connection of two independent channels is established through the \gls{tls} Handshake, from the client to the server and in the other direction from the server to the client. The Record Protocol is responsible for the data protection on these channels using the authenticated encryption scheme.\\
The responsibilities of the Record Protocol are:
\begin{itemize}
	\item fragmentation
	\item compression
	\item application of \gls{mac} and encryption
	\item transmisson 
\end{itemize}
But the tasks/responsibilities also depend on the parameters of the keys, namely how the keys were set during the handshake protocol. \\
The processing of data is illustrated and described below:      

\begin{figure}[H]
	\centering
		\includegraphics[scale=0.5]{images/tls_recordprotocol.png}
	\caption{\gls{tls} Record Protocol}
	\label{fig:tls_recordprotocol}
\end{figure}

First the Record Protocol layer receives the data from the application layer. Then it fragments the data to a size appropriate for the cryptographic algorithm. 
The next step is the compression, which is done if specified. The compression algorithm would then be defined in the current session state and would translate the date to a TLSCompressed structure.
 
Subsequently the message authentication code over the compressed data would be computed.
To ensure that the data were not altered during the transmission, the \gls{mac} value is added so the receiver can check if the incoming \gls{mac} value and the computed one 
match. Thus, the integrity and confidentiality are ensured.

What should be mentioned is, the first handshake is neither secured by a \gls{mac} nor encrypted. Because this is the initial value for the keys, it is how they are first set. After the first handshake these parameters are set and therefore secured and encrypted.
\cite{ms:Record} \cite{Hassenstein}