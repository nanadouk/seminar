\chapter*{Abstract}
\label{chap:abstract}
The Transport Layer Security (\gls{tls}) is used for secured and encripted comunication over networks. 2018 was the release of the new \gls{tls} 1.3 which is defined in RFC8446. In this study we are going to figure out, what is new, what the changes to \gls{tls} 1.2 exactly are and what is remaining. 

The questions of the particular improves and the consequences for example the configurations or security would be identified. It is also described what is removed. 

The first part of the study gives a brief overview about \gls{tls} and the different protocols. This part gives an overal explanation how the process of the securing and encryption of the communication works. 

The main part is the comparison of \gls{tls} 1.3 and \gls{tls} 1.2. This was released with the comparison of the changes. Namely the handhsake protocol, the session ressumption and the key derivation functions and the connection renegotiation were faced and analysed.

In the final part is the conclusion and a general overview due to the gathered information by the comparison.



