\chapter*{Abstract}
\label{chap:abstract}
The Transport Layer Security (\gls{tls}) is used for secured and encrypted communication over the networks. 2018 was the release of the new \gls{tls} 1.3 which is defined in RFC8446. In this study we are going to figure out, what is new, what the changes to \gls{tls} 1.2 exactly are and what has remained the same. 

The questions of the particular improvement and the consequences, for example of the configurations or security will be identified. What was removed will also be noted and explained.

The first part of the study will give a brief overview about \gls{tls} and the different protocols. This part will give an overall explanation of how the process of the securing and encryption of the communication works. 

The main part will be the comparison of \gls{tls} 1.3 and \gls{tls} 1.2. This would be realised with the comparison of the changes. Namely the handhsake protocol, the session ressumption and the key derivation functions and supported ciphersuites which will be faced and analysed.

The conclusion will give a general overview based on the gathered information from the comparison.



